\documentclass[conference]{IEEEtran}

\begin{document}

\title{Sample IEEE paper by quicklatex.blogspot.com}
\author
{\IEEEauthorblockN{Author First,Anthor Second}
\IEEEauthorblockA{Department name\\
University Name\\
State Zipcode\\
Email:}
}
\maketitle

\begin{abstract}
This document applies to version 1.7 and later of IEEEtran.
Prior versions do not have all of the features described here.IEEEtran will display the version number on the user��s console when a document using it is being compiled. The latest version of IEEEtran and its support files can be obtained from IEEE��s web site, or CTAN . This latter site may have some additional material, such as beta test versions and files related to non-IEEE uses of IEEEtran. See the IEEEtran homepage for frequently asked questions and recent news about IEEEtran. Complimentary to this document.
\end{abstract}

\section{Introduction}
\label{sec:intro} 
This document applies to version 1.7 and later of IEEEtran.
Prior versions do not have all of the features described here.IEEEtran will display the version number on the user��s console when a document using it is being compiled. The latest version of IEEEtran and its support files can be obtained from IEEE��s web site, or CTAN . This latter site may have some additional material, such as beta test versions and files related to non-IEEE uses of IEEEtran. See the IEEEtran homepage for frequently asked questions and recent news about IEEEtran. Complimentary to this document.This document applies to version 1.7 and later of IEEEtran.Prior versions do not have all of the features described here.IEEEtran will display the version number on the user��s console when a document using it is being compiled. The latest version of IEEEtran and its support files can be obtained from IEEE��s web site, or CTAN . This latter site may have some additional material, such as beta test versions and files related to non-IEEE uses of IEEEtran. See the IEEEtran homepage for frequently asked questions and recent news about IEEEtran. Complimentary to this document. Rest of the paper is organized as follows.Section~\ref{sec:meth}......

\section{Methodologies}
\label{sec:meth} 
This document applies to version 1.7 and later of IEEEtran.
Prior versions do not have all of the features described here.IEEEtran will display the version number on the user��s console when a document using it is being compiled. The latest version of IEEEtran and its support files can be obtained from IEEE��s web site, or CTAN . This latter site may have some additional material, such as beta test versions and files related to non-IEEE uses of IEEEtran. See the IEEEtran homepage for frequently asked questions and recent news about IEEEtran. Complimentary to this document.
This document applies to version 1.7 and later of IEEEtran.
Prior versions do not have all of the features described here.IEEEtran will display the version number on the user��s console when a document using it is being compiled. The latest version of IEEEtran and its support files can be obtained from IEEE��s web site, or CTAN . This latter site may have some additional material, such as beta test versions and files related to non-IEEE uses of IEEEtran. See the IEEEtran homepage for frequently asked questions and recent news about IEEEtran. Complimentary to this document.
This document applies to version 1.7 and later of IEEEtran.
Prior versions do not have all of the features described here.IEEEtran will display the version number on the user��s console when a document using it is being compiled. The latest version of IEEEtran and its support files can be obtained from IEEE��s web site, or CTAN . This latter site may have some additional material, such as beta test versions and files related to non-IEEE uses of IEEEtran. See the IEEEtran homepage for frequently asked questions and recent news about IEEEtran. Complimentary to this document.
This document applies to version 1.7 and later of IEEEtran.
Prior versions do not have all of the features described here.IEEEtran will display the version number on the user��s console when a document using it is being compiled. The latest version of IEEEtran and its support files can be obtained from IEEE��s web site, or CTAN . This latter site may have some additional material, such as beta test versions and files related to non-IEEE uses of IEEEtran. See the IEEEtran homepage for frequently asked questions and recent news about IEEEtran. Complimentary to this document.


\end{document}